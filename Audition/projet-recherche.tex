%------------------------------------------------
\begin{frame}{Projet d'Intégration (SequeL)}

Intégration des thématiques de recherche ``RL" et "\'Elicitation” dans l'équipe ``SequeL":
% pour projets à moyen terme:
\begin{block}

\begin{itemize}
	%\item  Planification dans des MDPs avec des recomposes imprécises 
	\item Étendre mes algorithmes au contexte de POMDPs.
	\item Les approches \textit{policy gradient et les méthodes d'arm bandit}
	peuvent bénéficier de l'analyse des récompenses inconnues obtenues avec l'élicitation et le Minimax regret. 
	%\item Élicitation or Minimax regret méthodes peut être étendus sur les approches \textit{policy gradient et les méthodes d'arm bandit}.
\end{itemize}
\end{block}
%%

 Thèmes de recherche dans SequeL traitant des  approches complémentaires:
\begin{block}

\begin{itemize}
	\item  Inverse RL, Spoken Dialogue Systems - RL (Olivier Pietquin, Bilal Piot)
	\item  Statistical Machine Learning - RL (Odalric-Ambrym Maillard )
	\item  Machine Mearning, Minimal feedback - RL (Michal Valko )
	\item Philippe Preux ()
	\end{itemize}
\end{block}
\end{frame}

%------------------------------------------------
\begin{frame}{Projet d'Intégration (SequeL)}
Intégration les thématiques de recherche ``TAL" et "Apprentissage profond” dans l'équipe ``SequeL":
% de CRIStAL pour projets à moyen terme:
\begin{block}{}
	\begin{itemize}
		\item Dialogue système (guesswhat?! game):
			\begin{itemize}
				\item Implémenter Deep q-network le MDP de guesswhat, parce que le nombres d'états et actions sont infinies \cite{strub}.
				\item Classification des utilisateurs par rapport de leurs politiques optimales et prédire la politique optimale de nouveaux utilisateurs entrant dans le système \cite{carrara}.  
			\end{itemize}
		\item De manière générale, le RL profond est une thématique qui m'intéresse particulièrement. 
	\end{itemize}
\end{block}

[Strub et al. 2017] F. STRUB , H. DE VRIES , J. MARY, B. PIOT, A. COURVILLE , O.
PIETQUIN. End-to-end optimization of goal driven and visually grounded dialogue
systems. IJCAI 2017
\vspace{0.2cm}
[Carrara et al. 2017] N. CARRARA , R. LAROCHE , O. PIETQUIN. Online learning and transfer for user adaptation in dialogue systems. SIGDIAL/SEMDIAL 2017.

\end{frame}

%------------------------------------------------
\begin{frame}{Collaboration avec Magnet}
Points de connexion avec l’équipe ``Magnet” :

\vspace{0.5cm}

Parmi les autres domaines de recherche en Machine Learning, ceux-là m'intéresse en particulier :

\begin{block}{}
\begin{itemize}
	\item Aurélien Bellet (optimisation pour machine learning)
	\item Marc Tommasi, Mikaela Keller (application de Machine Learning en TAL, extraction d'information automatique)
\end{itemize}
\end{block}

\end{frame}
