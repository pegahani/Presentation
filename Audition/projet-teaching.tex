%------------------------------------------------
\begin{frame}{Enseignements}

\vspace{-1cm}
J’ai enseigné différents cours au niveau
\imp{licence et master} en \imp{Informatique}, \imp{Mathématique} et \imp{Économie} aux Universités \imp{Paris Dauphine} et \imp{Paris XIII}.

J’ai effectué deux années de Monitorat pour des étudiants en Informatique à l’ \imp{IUT} de Paris XIII.


\begin{overlayarea}{\textwidth}{4.cm}
\only<1>{
\begin{block}{Enseignements ($\simeq 600$ heures)}
\begin{itemize}
\item "Precalculus", "Calculus I et II. Equations différentielles ordinaires. \imp{Statistiques}. Algèbre linéaire numérique.
\item \imp{Programmation} (C++). Programmation objet et objet avancée. Interface graphique.  
\item Introduction aux interfaces homme-machine (Java, swing).
 \item Administrateur système (Marionnet), Unix.
 \item \imp{Bases de données, Bases de données avancées} (Oracle).
\item Visual Basic Applications et Excel (\imp{responsable du cours}).
\end{itemize}
\end{block}
}
\only<2>{
\begin{block}{Encadrement de projet (Université de Caen)}
\begin{itemize}
 \item  Master 2 : Implémentation de Deep Q-network : le trading des cryptomonnaies
 \item Master 2 : Génération du texte avec une méthode de Deep Learning (LSTM)
 \item Licence 3: Les stages de développement de logiciels pour des entreprises.
\end{itemize}
\end{block}
}
\end{overlayarea}

%
\end{frame}


%\begin{frame}{Enseignements}
%%Enseignements avant et durant la Thèse
%\begin{block}{Enseignements avant le Doctorat (250 heures)}
%J’ai enseigné différents cours en \imp{mathématique et statistique} pendent deux ans dans deux universités en Iran aux étudiants
%de Licence en \imp{ingénierie, informatique, statistique et économie}.
%\begin{itemize}
%\item "Precalculus", "Calculus I et II. Equations différentielles ordinaires. Statistiques. Algèbre linéaire numérique.
%\end{itemize}
%\end{block}
%%
%\begin{block}{Enseignements pendant le Doctorat (128 heures)}
%J’ai effectué deux années de Monitorat pour des étudiants de 1ère
%et 2ème année en Informatique à l’ \imp{IUT} de Paris XIII.
%\begin{itemize}
%\item Programmation (C++). Programmation et administration des BDD (Postgresql). 
%\item Introduction aux interfaces homme-machine (Java, swing).
%\end{itemize}
%\end{block}
%%
%\end{frame}
%%------------------------------------------------
%%------------------------------------------------
%
%
%
%%------------------------------------------------
%%------------------------------------------------
%\begin{frame}{Enseignements}
%%{Projet d’enseignement (Licence et Master de MIASHS)}
%%Les besoins principaux pour le poste sont les suivants:\\
% 	%MIASHS: Licence et master en Informatique et humanité numérique : Programmation, algorithmique, bases de données, fouille de données, réseaux et les cours bureautique.  
%
%\begin{block}{Enseignements pendant la dernière année de Doctorat (192 heures)}
%J’ai effectué un an d’ATER au niveau
%\imp{licence et master} en \imp{Informatique} et en \imp{Mathématique} à %l'Institut Galilée de 
%\imp{Paris XIII} :
%\begin{itemize}
% \item Administrateur système (Marionnet), Unix
% \item Élément d'informatique, Programmation impérative, Interface graphique (C, GTK)
% \item Bases de données, Bases de données avancées (Oracle).
%\end{itemize}
%\end{block}
%%%%%%
%\begin{block}{Demi-ATER (100 heures)}
%J’ai effectué un Demi-ATER a l’Université Paris Dauphine au niveau licence en \imp{mathématiques, informatique et économie}. 
%\begin{itemize} 
%\item Programmation objet et objet avancée (Java).
%\item Visual Basic Applications et Excel (\imp{responsable du cours}).
%\end{itemize}
%%
%\end{block}
%\end{frame}
%%------------------------------------------------
%%------------------------------------------------

%------------------------------------------------
%------------------------------------------------
%\begin{frame}{Enseignements} 
%\begin{block}{Encadrement de projet à l'Université de Caen}
%\begin{itemize}
% \item  Master 2 : Implémentation de Deep Q-network : le trading des cryptomonnaies
% \item Master 2 : Génération du texte avec une méthode de Deep Learning (LSTM)
% \item Licence 3: Les stages de développement de logiciels pour des entreprises.
%\end{itemize}
%\end{block}
%
%
%\end{frame}
%------------------------------------------------
%------------------------------------------------


%------------------------------------------------
%------------------------------------------------
\begin{frame}{Projet d'enseignement}
À partir de l'expérience acquise durant les dernières années, je souhaite poursuivre mon expérience d’enseignement en fonction des besoins et en particulier dans les domaines  suivants :
\begin{block}{licence et master MIASHS}
\begin{itemize}
\item \textbf{Programmation}, \textbf{algorithmique}, \textbf{bases de données}, fouille de données, programmation créative.
\item \textbf{Bureautique}, \textbf{statistique}.  
\end{itemize}
\end{block}
	
\begin{block}{Master MDLS}
\begin{itemize}
\item Machine learning et data science (\textbf{apprentissage automatique}, fouille de données, \textbf{bases de données}.
\end{itemize}
\end{block}


\end{frame}

%	Les besoins principaux pour MDLS sont Machine Learning et Data Science:
