%------------------------------------------------
\begin{frame}{Enseignements}
%Enseignements avant et durant la Thèse
\begin{block}{Enseignements avant le Doctorat (250 heures)}
J’ai enseigné différents cours en \imp{mathematique et statistique} pen-
dent deux années et demie a deux universités en Iran aux etudiants
de Licence en \imp{ingénierie, informatique, statistique et économie}.
\begin{itemize}
\item Precalculus, Calculus I, Calculus II. Equations Différentielles Ordinaires. Statistiques. Algèbre Linéaire Numérique.
\end{itemize}
\end{block}
%
\begin{block}{Enseignements pendant le Doctorat (128 heurs)}
J’ai effectué deux années de Monitorat pour des étudiants de 1ère
et 2ème année en Informatique à l’ \imp{IUT} de Paris XIII.
\begin{itemize}
\item Programmation (C++). Programmation et administration des BDD (Postgresql). 
\item Introduction aux interfaces homme-machine (Java,swing).
\end{itemize}
\end{block}
%
\end{frame}
%------------------------------------------------
%------------------------------------------------



%------------------------------------------------
%------------------------------------------------
\begin{frame}{Enseignements}
%{Projet d’enseignement (Licence et Master de MIASHS)}
%Les besoins principaux pour le poste sont les suivants:\\
 	%MIASHS: Licence et master en Informatique et humanité numérique : Programmation, algorithmique, bases de données, fouille de données, réseaux et les cours bureautique.  

\begin{block}{Enseignements pendant la dernière année de Doctorat (192 heurs)}
J’ai effectué un ans d’ATER pour des étudiants de 
\imp{licence et master} en \imp{Informatique} et \imp{économie} à %l'Institut Galilée de 
\imp{Paris 13}.
\begin{itemize}
 \item Administrateur système (Marionnet), Unix.
 \item Élément d'informatique, Programmation impérative, Interface graphique (C, GTK).
 \item Bases de données, Bases de données avancées (Oracle).
\end{itemize}
\end{block}
%%%%%
\begin{block}{Demi-ATER (100 heurs)}
J’ai effectué un Demi-ATER a l’Université Paris Dauphine aux étu-
diants de licence en \imp{mathématiques, informatique et économie}. 
\begin{itemize} 
\item Programmation objet et objet avancée (Java).
\item Visual Basic Applications et Excel (\imp{responsable du cours}).
\end{itemize}
%
\end{block}
\end{frame}
%------------------------------------------------
%------------------------------------------------

%------------------------------------------------
%------------------------------------------------
\begin{frame}{Enseignements} 
\begin{block}{Encadrement de projet à l'Université de Caen}
\begin{itemize}
 \item  Master 2 : Implémentation de Deep Q-network : le trading des cryptomonnaies
 \item Master 2 : Génération du texte avec une méthode de Deep Learning (LSTM)
 \item Licence 3
\end{itemize}
\end{block}


\end{frame}
%------------------------------------------------
%------------------------------------------------


%------------------------------------------------
%------------------------------------------------
\begin{frame}{Projet d'enseignement}
À partir de l'expérience acquise dans les dernières années d'enseignement, je souhaite poursuivre mon expérience d’enseignement dans les sujets suivantes:
\begin{block}{licence et master MIASHS}
\begin{itemize}
\item \textbf{Programmation}, algorithmique, \textbf{bases de données}, \textbf{fouille de données}.
\end{itemize}
\end{block}
	
\begin{block}{Master MDLS}
\begin{itemize}
\item Machine learning et data science (\textbf{apprentissage automatique}, fouille de données, \textbf{bases de données}.
\end{itemize}
\end{block}

\begin{block}{Master en humanités numériques}
\begin{itemize}
\item \textbf{Programmation}, programmation créative, \textbf{bases de données}.
\end{itemize}
\end{block}

\end{frame}

%	Les besoins principaux pour MDLS sont Machine Learning et Data Science:
